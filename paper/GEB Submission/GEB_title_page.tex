\documentclass{article}

\usepackage{arxiv}

\usepackage[utf8]{inputenc} % allow utf-8 input
\usepackage[T1]{fontenc}    % use 8-bit T1 fonts
\usepackage{lmodern}        % https://github.com/rstudio/rticles/issues/343
\usepackage{hyperref}       % hyperlinks
\usepackage{url}            % simple URL typesetting
\usepackage{booktabs}       % professional-quality tables
\usepackage{amsfonts}       % blackboard math symbols
\usepackage{nicefrac}       % compact symbols for 1/2, etc.
\usepackage{microtype}      % microtypography
\usepackage{graphicx}

\title{Prevailing scenarios of functional change in anthropocene bird
and mammal communities}

\author{
    Kari E. A. Norman
   \\
    Université de Montréal \\
  Montréal, Quebec \\
  \texttt{\href{mailto:kari.norman@berkeley.edu}{\nolinkurl{kari.norman@berkeley.edu}}} \\
   \And
    Perry de Valpine
   \\
    University of California, Berkeley \\
  Berkeley, California \\
  \texttt{\href{mailto:pdevalpine@berkeley.edu}{\nolinkurl{pdevalpine@berkeley.edu}}} \\
   \And
    Carl Boettiger
   \\
    University of California, Berkeley \\
  Berkeley, California \\
  \texttt{\href{mailto:cboettig@berkeley.edu}{\nolinkurl{cboettig@berkeley.edu}}} \\
  }


% tightlist command for lists without linebreak
\providecommand{\tightlist}{%
  \setlength{\itemsep}{0pt}\setlength{\parskip}{0pt}}



\usepackage{lineno}
\linenumbers
\begin{document}
\maketitle


\begin{abstract}
Aim: Despite unprecedented environmental change due to anthropogenic
pressure, recent work has found increasing species turnover but no
overall trend in species diversity through time at the local scale.
Functional diversity provides a potentially powerful alternative
approach for understanding community composition by linking shifts in
species identity to mechanisms of ecosystem processes. Here we present
the first multi-taxa, multi-system analysis of functional change through
time.

Location: Global, with a North American focus

Time period: 1923-2014

Major taxa studied: Mammals, Birds

Methods: We paired thousands of bird and mammal assemblage time series
from the BioTIME database with existing trait data representative of a
species' functional role to reconstruct time series of functional
diversity metrics. Using generalized linear mixed models we estimated
general trends in those metrics and trends for individual studies.

Results: We found no overall trend in any functional diversity metric,
despite data replicating species-based patterns of constant richness
with increasing turnover. The lack of trend held even after correcting
for changes in species richness. At the study-level, we identified four
prevailing scenarios of species and functional change, which showed
links to the duration of the observation window.

Main Conclusions: General trends indicate that on the aggregate one type
of functional shift is not more prevelant than the other across many
taxa, biomes, and realms. At the study-level, there were also a
substantial number of time series exhibiting no species or functional
change, however the majority of studies showed a shift in a species or
functional metric. With no one prevailing scenario of change, it will be
critical to link change scenarios to drivers of change, particularly to
identify communities with capacity to resist drivers from those not
experiencing substantial pressure from a driver.
\end{abstract}


\hypertarget{data-availability}{%
\section{Data Availability}\label{data-availability}}

Code for the analyses in this chapter is archived on Zenodo at
\href{https://doi.org/10.5281/zenodo.5514334}{10.5281/zenodo.5514334}.

Data products are also archived on Zenodo at
\href{https://doi.org/10.5281/zenodo.6499442}{10.5281/zenodo.6499442}.

Original data sources are open access and available at their respective
providers.

\hypertarget{funding}{%
\section{Funding}\label{funding}}

Support for this work was provided by U.S. Department of Energy through
the Computational Sciences Graduate Fellowship (DE-FG02-97ER25308)
awarded to KEAN. CB acknowledges support from NSF CAREER Award \#1942280
\& NIFA project CA-B-INS-0162-H.

\bibliographystyle{unsrt}
\bibliography{refs.bib}


\end{document}
